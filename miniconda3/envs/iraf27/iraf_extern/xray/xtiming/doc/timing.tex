\input skeleton.tex

\def\version{\it Version 1.0 --- 1/26/89}
\def\chapter{\it Timing Analysis}

\baselineskip 24pt

\pageno=1001  % force no pagenumber by making pagenumber >999
\count100=\pageno

\font\bigfont = cmr10 scaled 1440
\bigfont

\centerline{}
\vskip 2.truein
\centerline{ROSAT Project Working Document of Science Specifications for the}
\centerline{Post Reduction Off-line Software (PROS)}
\centerline{Timing Analysis}
\vskip 24pt
\centerline{\version}
\vskip 2.5truein
\hbox{Prepared by: \hfil}
\vskip 24pt
\centerline{Smithsonian Astrophysical Observatory}
\centerline{High Energy Astrophysics Division}
\vfill\eject

\pageno=1001  % force no pagenumber by making pagenumber >999
\count100=\pageno

\centerline{}
\vskip 3.truein
This document describes algorithms for the
timing analysis package of the ROSAT/IRAF Post Reduction Off-line
Software (the PROS). Jeffrey McClintock coordinated the science
requirements for the PROS timing analysis, and he and other
scientists at SAO and GSFC wrote these specifications (see `science
contacts' for individual chapters).
These algorithms go beyond a basic analysis system, and exceed
the capabilities of the SAO {\it Einstein} general-user timing
programs.  A subset of these algorithms are being implemented
for ROSAT by Janet DePonte and Maureen Conroy.  This is
a `working document' which is under frequent revision.  Please check
with Diana Worrall that you have the latest version, and inform her of
errors.
\blankline
Note that version numbers and dates heading even-numbered pages refer
to the last date on which the content of the corresponding
chapter was significantly modified.

\vfill\eject


\baselineskip 18.pt

\pageno=-1
\count100=\pageno

\rm

\def\leaderfill{\leaders\hbox to 1em{\hss.\hss}\hfill}
\centerline{}\vskip 24pt
\bigslant
\centerline{Table of Contents}
\rm


\vskip 18pt
\line{I. Light Curves \leaderfill 1}
\line{ \quad 1 Overview \leaderfill 1}
\line{ \quad 2 Make a light curve - bin data \leaderfill 1}
\line{ \quad 3 Specific Examples of Plot/List files \leaderfill 2}
\line{ \quad 4 Notes \leaderfill 4}
\line{II. Periodogram Analysis \leaderfill 5}
\line{ \quad 1 Overview \leaderfill 5}
\line{ \quad 2 Input Data \leaderfill 5}
\line{ \quad 3 Input Options \leaderfill 5}
\line{ \quad 4 Output Data \leaderfill 5}
\line{ \quad 5 Output Display \leaderfill 5}
\line{ \quad 6 Full Description \leaderfill 5}
\line{ \quad 7 Summary \leaderfill 8}
\line{ \quad 8 References \leaderfill 9}
\line{III. Correlation Analysis \leaderfill 10}
\line{ \quad 1 Overview \leaderfill 10}
\line{ \quad 2 Input Data \leaderfill 10}
\line{ \quad 3 Input Options \leaderfill 10}
\line{ \quad 4 Output Data \leaderfill 10}
\line{ \quad 5 Output Display \leaderfill 10}
\line{ \quad 6 Full Description \leaderfill 10}
\line{ \quad 7 References \leaderfill 12}
\line{IV. Folding and Period Searches \leaderfill 13}
\line{ \quad 1 Overview \leaderfill 13}
\line{ \quad 2 Folding Data \leaderfill 13}
\line{ \quad 3 Period Search using epoch folding \leaderfill 14}
\line{ \quad 4 The Rayleigh test \leaderfill 14}
\line{ \quad 5 Required Output \leaderfill 15}
\line{ \quad 6 References \leaderfill 15}
\line{V. Period Searching via truncated Fourier series \leaderfill 16}
\line{ \quad 1 Overview \leaderfill 16}
\line{ \quad 2 Input Data \leaderfill 16}
\line{ \quad 3 Output Data \leaderfill 17}
\line{ \quad 4 Choice of period/frequency steps \leaderfill 18}
\line{ \quad 5 References \leaderfill 18}
\line{VI. Random Deviations \leaderfill 19}
\line{ \quad 1 Overview \leaderfill 19}
\line{ \quad 2 Full Description: BINVAL (Number vs. arrival time) \leaderfill 19}
\line{ \quad 3 Full Description: ARRIVL (Interval distribution) \leaderfill 20}
\line{ \quad 4 References \leaderfill 21}
\line{ \quad 5 Appendix: Detection of random fluctuations including data gaps \leaderfill 22}
\line{VII. QPO Analysis \leaderfill 24}
\line{ \quad 1 Overview \leaderfill 24}
\line{ \quad 2 Leiden QPO Software \leaderfill 25}
\line{ \quad 3 A possible outline for the ROSAT QPO Software \leaderfill 27}
\line{VIII. Shotnoise Analysis \leaderfill 35}
\line{ \quad 1 Introduction \leaderfill 35}
\line{ \quad 2 The Basic Idea of the Analysis Method \leaderfill 35}
\line{ \quad 3 Complications and the Actual Method: The Recipe \leaderfill 36}
\line{ \quad 4 Case for a large number of bins \leaderfill 38}
\line{ \quad 5 Error Analysis \leaderfill 38}
\line{ \quad 6 Simulating Binned Shot Noise with an Exponential Profile \leaderfill 38}
\line{ \quad 7 Limitations of the ``shot'' model analysis \leaderfill 41}
\line{ \quad 8 References \leaderfill 41}
\line{ \quad 9 Flow Chart for Shot Noise Analysis \leaderfill 43}
\line{IX. Flare Monitor \leaderfill 44}
\line{ \quad 1 Overview \leaderfill 44}
\line{ \quad 2 Input Data \leaderfill 44}
\line{ \quad 3 Procedure \leaderfill 44}
\line{ \quad 4 Output \leaderfill 45}
\line{ \quad 5 ``De-gapping'' \leaderfill 45}
\line{ \quad 6 Notes \leaderfill 45}
\line{X. Fractal/Attractor Dimension Time Series Analysis \leaderfill 47}
\line{ \quad 1 Overview \leaderfill 47}
\line{ \quad 2 Basic Idea \leaderfill 47}
\line{ \quad 3 The Method \leaderfill 47}
\line{ \quad 4 Why the Algorithm Works \leaderfill 49}
\line{ \quad 5 Notes \leaderfill 49}
\line{ \quad 6 References \leaderfill 50}

\vfill\eject

%%%%%%%%%%%%%%%%%%%%%%%%%%%%%%%%%%%%%%%%%%%%%%%%%%%%%%%%%%%%%%%%%%%
%                                                                 %
%                     D.M.W.           January 1988               %
%                                                                 %
%%%%%%%%%%%%%%%%%%%%%%%%%%%%%%%%%%%%%%%%%%%%%%%%%%%%%%%%%%%%%%%%%%%
%This document uses macros defined in skeleton.tex.               %
%                                                                 %
%      Automatic numbering of sections goes as follows:           %
%               titles:  \@{title}                                %
%            subtitles:  \**subsections (item form)               %
%         subsubtitles:  \??subsubsections (itemitem form)        %
%                                                                 %
% For unnumbered lists following \** use:                         %
%                   {\list                                        %
%                                                                 %
%                   My first in list; Note that blank lines are   %
%                                                                 %
%                   My second in list;   important                %
%                                                                 %
%                   }                                             %
%                                                                 %
% For unnumbered lists following \?? use:                         %
%                   {\listlist                                    %
%                                                                 %
%                   same idea as above                            %
%                                                                 %
%                   }                                             %
%                                                                 %
%Equations in display mode ($$   $$) should lie outside lists and %
%                                             listlists.          %
% Tables need to \moveright 2\parindent or \moveright 3\parindent %
%                                                                 %
%%%%%%%%%%%%%%%%%%%%%%%%%%%%%%%%%%%%%%%%%%%%%%%%%%%%%%%%%%%%%%%%%%%
\baselineskip 12.pt
\pageno=1
\count100=\pageno
\input ltcurv.tex
\startcount
\input period.tex
\startcount
\input correlate.tex
\startcount
\input fold.tex
\startcount
\input fourier.tex
\startcount
\input random.tex
\startcount
\input qpo.tex
\startcount
\input shot.tex
\startcount
\input flaremon.tex
\startcount
\input fractal.tex
\vfill\eject\end
