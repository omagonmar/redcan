% This is the beginning of skeleton.tex
%    Version to be used with PROS science specifications
%%%%%%%%%%%%%%%%%%%%%%%%%%%%%%%%%%%%%%%%%%%%%%%%%%%%%%%%%%%%%
%  Author D.M.W.                                            %
%%%%%%%%%%%%%%%%%%%%%%%%%%%%%%%%%%%%%%%%%%%%%%%%%%%%%%%%%%%%%
%
\hsize 6.5truein
\vsize 9.0truein
\hoffset=0.0truein
\voffset=0.0truein

%input font definitions for pica typeface.
%
\font\tenrm  =cmr10  scaled 1200
\font\sevenrm=cmr7   scaled 1200
\font\fiverm =cmr5   scaled 1200
\font\teni   =cmmi10 scaled 1200
\font\seveni =cmmi7  scaled 1200
\font\fivei  =cmmi5  scaled 1200
\font\tensy  =cmsy10 scaled 1200
\font\sevensy=cmsy7  scaled 1200
\font\fivesy =cmsy5  scaled 1200
\font\tenex  =cmex10 scaled 1200
\font\tenbf  =cmbx10 scaled 1200
\font\sevenbf=cmbx7  scaled 1200
\font\fivebf =cmbx5  scaled 1200
\font\tensl  =cmsl10 scaled 1200
\font\tentt  =cmtt10 scaled 1200
\font\tenit  =cmti10 scaled 1200
%
% (3) Set up family trees
%
\catcode`\@=11
% 
% (3.1) Family 0: text (\rm from \tenrm, \sevenrm, \fiverm)
%
\textfont0=\tenrm \scriptfont0=\sevenrm \scriptscriptfont0=\fiverm
\def\rm{\fam\z@\tenrm}
%
% (3.2) Family 1: maths (\mit from \teni, \seveni, \fivei)
%
\textfont1=\teni  \scriptfont1=\seveni  \scriptscriptfont1=\fivei
\def\mit{\fam\@ne} 
\def\oldstyle{\fam\@ne\teni}
%
% (3.3) Family 2: maths symbols (\cal from \tensy, \sevensy, \fivesy)
%
\textfont2=\tensy \scriptfont2=\sevensy \scriptscriptfont2=\fivesy
\def\cal{\fam\tw@}
%
% (3.4) Family 3: maths symbols praticularly integrals (from \tenex)
%
\textfont3=\tenex \scriptfont3=\tenex   \scriptscriptfont3=\tenex
%
% (3.5) Family 4: italics (\it from \tenit)
%
\newfam\itfam \def\it{\fam\itfam\tenit} 
\textfont\itfam=\tenit
%
% (3.6) Family 5: slanted (\sl from \tensl)
%
\newfam\slfam \def\sl{\fam\slfam\tensl} 
\textfont\slfam=\tensl
%
% (3.7) Family 6: bold (\bf from \tenbf, \sevenbf, \fivebf)
%
\newfam\bffam \def\bf{\fam\bffam\tenbf} 
\textfont\bffam=\tenbf \scriptfont\bffam=\sevenbf
\scriptscriptfont\bffam=\fivebf
%
% (3.8) Family 7: typewriter (\tt from \tentt)
%
\newfam\ttfam \def\tt{\fam\ttfam\tentt} 
\textfont\ttfam=\tentt
%
\catcode`\@=12
\font\bigslant =cmsl10 scaled 1728
%
% Useful definitions related to paragraphs, blank lines and references.
%
\def\blankline{\par\vskip 12 pt\noindent}
\def\newline{\par\noindent}
\def\blankpar{\par\vskip 12 pt}
\def\refindent{\noindent\hangafter=1 \hangindent 24 pt}
\def\listlist{\advance\leftskip by1.5\parindent
  \everypar={\hangafter=1 \hangindent 54.0pt}}
\def\list{\advance\leftskip by0.5\parindent
  \everypar={\hangafter=1 \hangindent 36.0pt}}

%set up pagenumbers useful for memos and letters.  This is superceeded
%  later by the headline definition.  Headline may be commented out.
%
\def\folio{\ifnum\pageno=1\nopagenumbers\else\number\pageno\fi}

% line spacing set at 1.0, parskip to 1/2:
%
\baselineskip 12.0pt
\parskip 8.0pt
\parindent 36.0pt

\interlinepenalty=1000
\raggedbottom
\pretolerance 10000
\hyphenpenalty 10000

% input definitions of \simless, \simgreat and \simpropto for
% approximately less than, greater than and proportional to:
%
\newbox\grsign \setbox\grsign=\hbox{$>$} \newdimen\grdimen \grdimen=\ht\grsign
\newbox\simlessbox \newbox\simgreatbox
\setbox\simgreatbox=\hbox{\raise.5ex\hbox{$>$}\llap
     {\lower.5ex\hbox{$\sim$}}}\ht1=\grdimen\dp1=0pt
\setbox\simlessbox=\hbox{\raise.5ex\hbox{$<$}\llap
     {\lower.5ex\hbox{$\sim$}}}\ht2=\grdimen\dp2=0pt
\def\simgreat{\mathrel{\copy\simgreatbox}}
\def\simless{\mathrel{\copy\simlessbox}}
% Next lines define ``approximately proportional to''
\newbox\simppropto
\setbox\simppropto=\hbox{\raise.5ex\hbox{$\sim$}\llap
     {\lower.5ex\hbox{$\propto$}}}\ht2=\grdimen\dp2=0pt
\def\simpropto{\mathrel{\copy\simppropto}}

% set up some definitions which will allow automatic numbering of
%\item and \itemitem
%To get titles, use \@{title}
%To get subtitles use \**subtitles here
%To get subsubtitles use \??
%
%
\def\startcount{\global\count1=0 \global\count2=0 \global\count3=0}
\def\numitemitem{\number\count1.\number\count2.\number\count3}
\def\numitem{\number\count1.\number\count2}
\def\num{\refindent{\bf \number\count1.~}}
\def\addnum{\global\advance\count1 by 1 \global\count2=0
             \global\count3=0 \blankline}
\def\addsubnum{\global\advance\count2 by 1 \global\count3=0 }
\def\addsubsubnum{\global\advance\count3 by 1 }
\def\@#1{\leftskip +0pt \addnum \num{\bf #1}}
\def\**{\leftskip +10pt \addsubnum \item{\numitem}}
\def\??{\leftskip +20pt \addsubsubnum \itemitem{\numitemitem}}

%Here we will set up the headline.
%  It will consist of the page number, version number 
%  and the chapter.
%
\nopagenumbers
\headline={\ifnum\pageno=\count100 \blankheadline \else 
           {\ifodd\pageno\rightheadline \else\leftheadline\fi}\fi}
\def\rightheadline{\hfil \chapter \hfil
  {\bf{\ifnum\pageno<0 \romannumeral-\pageno \else\number\pageno\fi}}}
\def\leftheadline{
  {\bf{\ifnum\pageno<0 \romannumeral-\pageno \else\number\pageno\fi}}
          \hfil\version} 
\def\blankheadline{\ifnum\pageno<1000 {
          \ifodd\pageno\rightheadnum \else\leftheadnum\fi} \else \hfill\fi}
\def\rightheadnum{\hfil {\bf
    {\ifnum\pageno<0 \romannumeral-\pageno \else\number\pageno\fi}}}
\def\leftheadnum{{\bf 
    {\ifnum\pageno<0 \romannumeral-\pageno \else\number\pageno\fi}}\hfil}
\pageno=1
\count100=\pageno
\voffset=2\baselineskip
%
%  We will set initial values of version and chapter to blank.
\def\version{}
\def\chapter{}

%Define chapterhead for Chapter title
%
\def\chapterhead#1{\centerline{}\vskip 12pt \bigslant
                   \centerline{#1} \rm}

%now set the alignment for equations:
\def\leftdisplay#1$${\leftline{\indent\indent\indent$\displaystyle{#1}$}$$}
\everydisplay{\leftdisplay}

%Finally, make today's date available
%
\def\today{\ifcase\month\or
  January\or February\or March\or April\or May\or June\or
  July\or August\or September\or October\or November\or December\fi
  \space\number\day, \number\year}

\rm

% Start counter
\startcount

%This is the end of skeleton.tex
% Specific text comes in here.  End document with /vfill/eject/end.
%%%%%%%%%%%%%%%%%%%%%%%%%%%%%%%%%%%%%%%%%%%%%%%%%%%%%%%%%%%%%%%%%%%%%




